
\section{Introducción}

Los sistemas de información geográfica (SIG), son herramientas que nos permiten
trabajar con diferentes tipos de datos geo-espaciales \footnote{Un dato
geoespacial describe un objeto u evento en la superficie de la tierra. Se
conforma de dos coordenadas terrestres y un valor nominal (carácterística del
objeto o evento en cuestión).}, por ejemplo datos vectoriales y datos ráster.

Los \textit{datos vectoriales} son datos de tipo punto, línea y/o polígono que
representan características como propiedades, ciudades, carreteras, montañas o
cuerpos de agua.

Los \textit{datos ráster} son céldas pixeladas o cuadrículadas que se
identifican según la fila y la columna. Estos datos crean imágenes más complejas
como fotografías e imágenes de satélite.

Los SIG, además de ayudarnos a visualizar los datos geo-espaciales, nos permiten
hacer análisis con el fin de gestionar acciones, dar acciones a emergencias,
establecer prioridades, comprender tendencias y generar mapas e informes.
