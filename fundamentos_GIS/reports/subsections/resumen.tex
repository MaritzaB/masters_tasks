
\section{Resumen}

Para esta práctica, trabajamos con datos de trayectorias de Albatros de Layssan,
que son una especie de ave marina que actualmente se encuentra amenazada
\footnote{De acuerdo a la NORMA Oficial Mexicana NOM-059-SEMARNAT-2010, aquellas
que podrían llegar a encontrarse en peligro de desaparecer a corto o mediano
plazo, si siguen operando los factores que inciden negativamente en su
viabilidad, al ocasionar el deterioro o modificación de su hábitat o disminuir
directamente el tamaño de sus poblaciones.} y cuyo lugar de anidación es en Isla
Guadalupe.

El albatros de Layssan como ave marina, pasa la mayor parte del tiempo en el mar
(a menudo más del 90\%) y solo toca tierra durante la étapa reproductiva.

En esta practica analizamos sus trayectorias anuales y el área en la que se
desplazan durante su etapa reproductiva. El periódo de reproducción va desde el
1 de diciembre hasta el 30 de julio, y está dividida de la siguiente manera:

\begin{table}[h!]
\caption{Etapas reproductivas del Albatros de Layssan.}
\begin{center}
\begin{tabular}{lcc}
    Etapa & Fecha de inicio & Fecha de término \\
    \hline
    Incubación & 1 de diciembre & 6 de febrero \\
    Empolle & 7 de febrero & 20 de febrero \\
    Crianza & 21 de febrero & 30 de julio
\end{tabular}
\end{center}
\end{table}
