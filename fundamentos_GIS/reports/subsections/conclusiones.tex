\section{Conclusiones}

\begin{itemize}
    \item Las herramientas SIG permiten una amplia variedad de herramientas para
    procesar datos y analizarlos.
    \item Los bases de datos geoespacial son muy útiles para manejar grandes
    cantidades de datos y poder hacer operaciones. Además nos permiten
    reproducir el procesamiento a través del código SQL.
    \item Gracias a las consultas a la base de datos de PostgreSQL pudimos
    clasificar los datos de las trayectorias de albatros en sus diferentes
    temporadas rerpoductivas.
    \item Usamos varias funciones de PostGIS para obtener la envolvente convexa
    y visualizar que tan amplias son las áreas de desplazamiento de los
    albatros.
    \item Otra herramienta utilizada para afinar nuestro procesamiento fue la
    diferencia geométrica de los polígonos. Así, eliminamos una pequeña parte de
    área que era por seguro que no formaba parte de las trayectorias.
    \item Todo este procesamiento nos permitió calcular las áreas de
    desplazamiento aproximado.
    \item Como un pequeño análisis, nos pudimos dar cuenta que la temporada en
    la que menor área de desplazamientocubrían las aves es durante el
    empollamiento. Durante las otras dos etapas reproductivas el área que
    recorren es semejante.
\end{itemize}
