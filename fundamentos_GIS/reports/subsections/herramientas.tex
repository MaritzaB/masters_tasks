
\subsection{Herramientas utilizadas en el proyecto}

Las herramientas utilizadas en este proyecto son QGIS, PostgreSQL y PostGIS.

QGIS o Quantum GIS es un software de código abierto que permite la
visualización, edición y análisis de datos geoespaciales. 

PostgreSQL es un sistema de gestión de bases de datos relacional de código
abierto que permite almacenar grande volúmenes de datos de manera eficiente y
segura.

PostGIS es una extensión espacial para PostgreSQL que agrega soporte para los
datos geoespaciales. Con esta extensión es posible almacenar datos
geoespaciales. A su vez, PostGIS también cuenta con funciones implementadas para
realizar operaciones geoespaciales como el cambio de sistema de referencia
cartesiano (\texttt{ST\_transform}) y operaciones topológicas como las
siguientes, por mencionar algunas:

\begin{itemize}
\item Intersección (\texttt{ST\_Intersection}): Encuentra la geometría resultante de la
intersección entre dos o más objetos espaciales.

\item Unión (\texttt{ST\_Union}): Combina las geometrías de dos o más objetos espaciales en un
único objeto resultante.

\item Superposición (\texttt{ST\_Overlap}): Verifica si dos geometrías se superponen en alguna
medida.

\item Buffer (\texttt{ST\_Buffer}) Crea un área o zona de influencia alrededor de una geometría, basada en
una distancia específica.

\item Longitud (\texttt{ST\_Length}): Calcula la longitud de una línea o el perímetro de un
polígono.

\item Área (\texttt{ST\_Area}): Calcula el área de un polígono.

\end{itemize}