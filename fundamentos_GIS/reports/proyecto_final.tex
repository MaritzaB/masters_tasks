% LaTeX Template for short student reports. Citations should be in bibtex format
% and go in references.bib
\documentclass[a4paper, 11pt]{article}
\usepackage[top=3cm, bottom=3cm, left = 2cm, right = 2cm]{geometry} 
\geometry{a4paper} 
\usepackage[T1]{fontenc}
\usepackage{textcomp}
\usepackage{graphicx} 
\usepackage{amsmath,amssymb}  
\usepackage{bm}  
\usepackage[pdftex,bookmarks,colorlinks,breaklinks]{hyperref}  
%\hypersetup{linkcolor=black,citecolor=black,filecolor=black,urlcolor=black} %
%black links, for printed output
\usepackage{memhfixc} 
\usepackage{pdfsync}  
\usepackage{fancyhdr}
\pagestyle{fancy}
\usepackage[spanish, es-tabla]{babel}
\usepackage{color}
\usepackage{upquote,listings}
\usepackage{pdflscape}


\definecolor{codegreen}{rgb}{0,0.6,0}
\definecolor{codegray}{rgb}{0.5,0.5,0.5}
\definecolor{codepurple}{RGB}{219, 48, 122}
\definecolor{backcolour}{RGB}{242, 242, 242}
\definecolor{bookColor}{cmyk}{0,0,0,0.90}  
\color{bookColor}

\lstset{upquote=true}

\lstdefinestyle{mystyle}{
    backgroundcolor=\color{backcolour},   
    commentstyle=\color{codegreen},
    keywordstyle=\color{codepurple},
    numberstyle=\numberstyle,
    stringstyle=\color{codepurple},
    basicstyle=\footnotesize\ttfamily,
    breakatwhitespace=false,
    breaklines=true,
    captionpos=b,
    keepspaces=true,
    numbers=left,
    numbersep=10pt,
    showspaces=false,
    showstringspaces=false,
    showtabs=false,
}
\lstset{style=mystyle}

\newcommand\numberstyle[1]{%
    \footnotesize
    \color{codegray}%
    \ttfamily
    \ifnum#1<10 0\fi#1 |%
}


\title{ Análisis geo-espacial de datos de gps de trayectorias de Albatros de
Layssan. \\
Proyecto final de fundamentos de Sistemas de Información Geográfica \\
}
\author{Ana Maritza Bello Yáñez \\ Profesora: Dra. Magdalena Saldaña}
%\date{}

\begin{document}
\maketitle
\tableofcontents

\section{Objetivos del proyecto}

Se requiere de un proyecto que englobe los siguientes elementos:

$\text{\rlap{$\checkmark$}}\square$ Creación de una base de datos sobre un tema
en concreto.

$\text{\rlap{$\checkmark$}}\square$ Base de datos con al menos dos tablas.

$\text{\rlap{$\checkmark$}}\square$ Representación cartográfica derivada de
algún elemento de la base de datos.

$\text{\rlap{$\checkmark$}}\square$ Aplicar al menos dos elementos de
geo-procesamiento.



\section{Introducción}

Los sistemas de información geográfica (SIG), son herramientas que nos permiten
trabajar con diferentes tipos de datos geo-espaciales.

Los SIG además de ayudarnos a visualizar los datos geo-espaciales, nos permiten
hacer análisis con el fin de gestionar acciones, dar acciones a emergencias,
establecer prioridades, comprender tendencias y generar mapas e informes.

Para esta práctica, estaremos trabajando con datos de trayectorias de Albatros
de Layssan, que son una especie de ave marina que actualmente se encuentra
amenazada \footnote{De acuerdo a la NORMA Oficial Mexicana
NOM-059-SEMARNAT-2010, aquellas que podrían llegar a encontrarse en peligro de
desaparecer a corto o mediano plazo, si siguen operando los factores que inciden
negativamente en su viabilidad, al ocasionar el deterioro o modificación de su
hábitat o disminuir directamente el tamaño de sus poblaciones.} y cuyo lugar de
anidación es en Isla Guadalupe.

El albatros de Layssan como ave marina, pasa la mayor parte del tiempo en el mar
(a menudo más del 90\%) y solo toca tierra durante la étapa reproductiva.

En esta practica estaremos analizando sus trayectorias anuales y el área en la
que se desplazan durante la étapa reproductiva. El periódo de étapa reproductiva
va desde el 1 de diciembre hasta el 30 de julio, y está dividida de la siguiente
manera:


\begin{table}[h!]
\caption{Étapas reproductivas del Albatros de Layssan.}
\begin{center}
\begin{tabular}{lcc}
    Etapa & Fecha de inicio & Fecha de término \\
    \hline
    Incubación & 1 de diciembre & 6 de febrero \\
    Empolle & 7 de febrero & 20 de febrero \\
    Crianza & 21 de febrero & 30 de julio
\end{tabular}
\end{center}
\end{table}


\subsection{Herramientas utilizadas}

\subsubsection{QGIS}
QGIS es un Sistema de Información Geográfica profesional de código abierto, que
posibilita la cración, visualización, análisis, edición y publicación de
información geoespacial.

Algunas de las ventajas que nos ofrece QGIS, y que estaremos utilizando
particularmente para esta practica son:

\begin{itemize}
    \item Soporte para la extensión espacial de PostgreSQL, PostGIS.
    \item Manejo de archivos vectoriales shapefile.
    \item Soporte para archivos raster.
\end{itemize}

\subsubsection{Postgress}

\subsubsection{Postgis}

\section{Resumen}


\section{Desarrollo}

\subsection{Creación de la base de datos de Trayectorias de Albatros de Layssan}

\lstinputlisting[language=SQL,   
framesep=10pt, framextopmargin=10pt] {../src/create_table_albatross.sql}

\section{Conclusiones y trabajo a futuro}


% \bibliographystyle{abbrv} \bibliography{references}  % need to put bibtex
% references in references.bib 
\end{document}
