% LaTeX Template for short student reports.
% Citations should be in bibtex format and go in references.bib
\documentclass[a4paper, 11pt]{article}
\usepackage[top=3cm, bottom=3cm, left = 2cm, right = 2cm]{geometry} 
\geometry{a4paper} 
\usepackage[T1]{fontenc}
\usepackage{textcomp}
\usepackage{graphicx} 
\usepackage{amsmath,amssymb}  
\usepackage{bm}  
\usepackage[pdftex,bookmarks,colorlinks,breaklinks]{hyperref}  
%\hypersetup{linkcolor=black,citecolor=black,filecolor=black,urlcolor=black} % black links, for printed output
\usepackage{memhfixc} 
\usepackage{pdfsync}  
\usepackage{fancyhdr}
\pagestyle{fancy}
\usepackage[spanish, es-tabla]{babel}

\title{Sample Report}
\author{Author McWriterson}
%\date{}

\begin{document}
\maketitle
\tableofcontents

\section{Introducción}

My project was about \ldots

I developed a system to \ldots

We did some experiments to find out \ldots

The main results were \ldots

\pagebreak

\section{Resumen}

I did some background reading in the following areas \ldots


\pagebreak

\section{Objetivos del proyecto}

Se requiere de un proyecto que englobe dos elementos principales, una base de
datos y una representación cartográfica derivada de algún elemento de la base de
datos.

Los datos deben representar información sobre un tema en concreto y parte de
ellos debe provenir de la base de datos. La base debe tener al menos dos tablas.

A partir de los datos se generará una representación cartográfica en la que se
apliquen al menos dos elementos de geo-procesamiento de los vistos en clase.

\pagebreak

\section{Desarrollo}

\section{Conclusiones y trabajo a futuro}

From our experiments we can conclude that \ldots

% \bibliographystyle{abbrv}
% \bibliography{references}  % need to put bibtex references in references.bib 
\end{document}
