\documentclass[12pt]{article}
\usepackage[utf8]{inputenc}
\usepackage[english]{babel}
 
\usepackage{multicol}
\usepackage[a4paper]{geometry}
\geometry{top=1.5cm, bottom=1.0cm, left=1.25cm, right=1.25cm}
 
\begin{document}
\begin{multicols}{2}
    
\section{Ética en la ingeniería de software}

¿Qué es?

¿Para qué sirve?

¿Dónde y como aplicarla?

La ética en el ámbito profesional se refiere al conjunto de normas y valores que
mejoran el desarrollo de las actividades profesionales. Ahora, en el ámbito de
la ingeniería de software, se refiere 

Like other engineering disciplines, software engineering is carried out within a
social and legal framework that limits the freedom of people working in that
area. As a software engineer, you must accept that your job involves wider
responsibilities than simply the application of technical skills. You must also
behave in an ethical and morally responsible way if you are to be respected as a
professional engineer. It goes without saying that you should uphold normal
standards of honesty and integrity. You should not use your skills and abilities
to behave in a dishonest way or in a way that will bring disrepute to the
software engineering profession. However, there are areas where standards of
acceptable behavior are not bound by laws but by the more tenuous notion of
professional responsibility. Some of these are:

Confidentiality You should normally respect the confidentiality of your
employers or clients regardless of whether or not a formal confidentiality
agreement has been signed. Competence You should not misrepresent your level of
competence. You should not knowingly accept work that is outside your
competence. Intellectual property rights You should be aware of local laws
governing the use of intellectual property such as patents and copyright. You
should be careful to ensure that the intellectual property of employers and
clients is protected. Computer misuse You should not use your technical skills
to misuse other people's computers. Computer misuse ranges from relatively
trivial (game playing on an employer's machine) to extremely serious
(dissemination of viruses or other malware). Professional societies and
institutions have an important role to play in setting ethical standards.
Organizations such as the ACM, the IEEE (Institute of Electrical and Electronic
Engineers), and the British Computer Society publish a code of professional
conduct or code of ethics. Members of these organizations undertake to follow
that code when they sign up for membership. These codes of conduct are generally
concerned with fundamental ethical behavior.

\end{multicols}
\end{document}
