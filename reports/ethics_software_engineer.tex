\documentclass[twoside]{article}

\usepackage{lipsum} % Package to generate dummy text throughout this template

\usepackage[sc]{mathpazo} % Use the Palatino font
\usepackage[T1]{fontenc} % Use 8-bit encoding that has 256 glyphs
\linespread{1.05} % Line spacing - Palatino needs more space between lines
\usepackage{microtype} % Slightly tweak font spacing for aesthetics

\usepackage[hmarginratio=1:1,top=32mm,columnsep=20pt]{geometry} % Document margins
\usepackage{multicol} % Used for the two-column layout of the document
\usepackage[hang, small,labelfont=bf,up,textfont=it,up]{caption} % Custom captions under/above floats in tables or figures
\usepackage{booktabs} % Horizontal rules in tables
\usepackage{float} % Required for tables and figures in the multi-column environment - they need to be placed in specific locations with the [H] (e.g. \begin{table}[H])
\usepackage{hyperref}    % Vinculos [que no estén subrayados]
\hypersetup{
    colorlinks=true,
    linkcolor=magenta,
    urlcolor=blue,
    citecolor=black
    }

\usepackage{lettrine} % The lettrine is the first en\usepackage[hidelinks]{hyperref}    % Vinculos [que no estén subrayados]larged letter at the beginning of the text
\usepackage{paralist} % Used for the compactitem environment which makes bullet points with less space between them

\usepackage{abstract} % Allows abstract customization
\renewcommand{\abstractnamefont}{\normalfont\bfseries} % Set the "Abstract" text to bold
\renewcommand{\abstracttextfont}{\normalfont\small\itshape} % Set the abstract itself to small italic text

\usepackage{titlesec} % Allows customization of titles
\renewcommand\thesection{\Roman{section}} % Roman numerals for the sections
\renewcommand\thesubsection{\Roman{subsection}} % Roman numerals for subsections
\titleformat{\section}[block]{\large\scshape\centering}{\thesection.}{1em}{} % Change the look of the section titles
\titleformat{\subsection}[block]{\large}{\thesubsection.}{1em}{} % Change the look of the section titles

\usepackage{fancyhdr} % Headers and footers
\pagestyle{fancy} % All pages have headers and footers
\fancyhead{} % Blank out the default header
\fancyfoot{} % Blank out the default footer
\fancyhead[C]{Ética en la ingeniería de Software $\bullet$ Noviembre 2022}
\fancyfoot[RO,LE]{\thepage} % Custom footer text

%----------------------------------------------------------------------------------------
%	TITLE SECTION
%----------------------------------------------------------------------------------------

\title{\vspace{-15mm}\fontsize{24pt}{10pt}\selectfont\textbf{Ética en la
ingeniería de Software}}

\author{
\large
\textsc{Ana Maritza Bello Yáñez} \\
\normalsize Centro de Investigación en Computación \\ 
\normalsize {abelloy2022@cic.ipn.mx}
\vspace{-5mm}
}
\date{}

%----------------------------------------------------------------------------------------

\begin{document}

\maketitle % Insert title

\thispagestyle{fancy} % All pages have headers and footers

%----------------------------------------------------------------------------------------
%	ABSTRACT
%----------------------------------------------------------------------------------------

\begin{abstract}

\noindent Este trabajo aborda aspectos relacionados a la ética en la Ingeniería
de Software. También muestra los esfuerzos realizados por diferentes
instituciones por definir un estándar de ética para profesionales de la
ingeniería de software. Los códigos analizados para este trabajo son el
\textit{SECODE} implementado por la \textit{IEEE/ACM}, y algunos análisis hechos
por diferentes autores acerca del impacto que podemos generar en la sociedad.

\end{abstract}

%----------------------------------------------------------------------------------------
%	ARTICLE CONTENTS
%----------------------------------------------------------------------------------------

\begin{multicols}{2} % Two-column layout throughout the main article text

\section{Introducción}

La ingeniería de software es una disciplina que combina las ciencias de la
computación, ciencias aplicadas y ciencias básicas basándose en la ingeniería.
Esta disciplina busca apoyar al desarrollo de software profesional. Cuando
hablamos de software no nos referimos solamente a programas, sino a toda la
documentación  asociada y a los datos de configuración requerido para hacer que
los programas funcionen de manera correcta \cite{sommerville2005ingenieria}.

La ética profesional, a su vez, consiste en un conjunto de saberes, creencias,
normas y valores que rigen el actuar de las personas en el campo profesional, y
en este caso dentro del campo de desarrollo de software
\cite{yuren2013etica}.

Así al hablar, de ética en el desarrollo de software, estamos considerando todos
aquellos aspectos relacionados a cómo desempeñamos nuestra profesión como
ingenieros de software.

Es importante destacar que no debemos confundir la ética profesional con el
cumplimiento de una ley.

\section{¿Qué es la ingeniería de software?}

La ingeniería de software, es una disciplina de la ingeniería que se interesa
por todos los aspectos de la producción de software, con el objetivo de que el
producto final (programas de cómputo y documentación asociada) tenga un
funcionamiento correcto. Es decir, que se le entregue al usuario un software con
la funcionalidad y desempeño requeridos, además de que sea confiable y
utilizable.

La ingeniería de software es una ciencia aplicada. Todos los productos de la
ingeniería de software involucran personas, por lo que en cualquier etapa del
desarrollo del producto se debe tener en cuenta a los usuarios de los productos
intermedios y finales. Los ingenieros de software tienen obligaciones con los
usuarios de sus productos, que incluyen no solo el sistema implementado, sino
también otros productos, como requisitos, planes de gestión de proyectos de
software, especificaciones, diseños, documentación, conjuntos de pruebas,
programas, manuales de usuario y materiales de capacitación.

Al desarrollar estos productos de ingeniería de software, cada decisión es un
compromiso que se ve afectado por restricciones tales como el presupuesto
disponible, las necesidades de los clientes, el software disponible, los
requisitos de confiabilidad, las consideraciones ambientales, los efectos
sociales e incluso las realidades políticas. Todo esto hace que el trabajo del
ingeniero de software profesional sea más difícil y requiera un juicio más
subjetivo. Pero hay algunas pautas para hacer estos juicios.

\section{¿Dónde y como aplicar la ética en la ingeniería de software?}

El software está presente en la vida diaria de las personas, y muchas de ellas
dependen del software. Entonces, la pregunta es: ¿Cómo nos impacta esto? ó mejor
dicho, como ingenieros de software: ¿Qué impacto tienen nuestras acciones en la
sociedad?

¿Qué nos asegura que el software que estamos usando fue construido de la mejor
manera?

Por supuesto que nos gustaría saber que el software de nuestros automoviles fue
bien desarrollado al igual que el software de los aviones en los que volamos.

Otras preguntas que nos conciernen respecto a este tema son:

¿Quién tiene la culpa si un carro o avión, cuyo sistema es autónomo, tiene un
accidente?

¿Quién se aseguró de que ese sistema es de calidad?

¿Qué es el software de calidad?, y ¿qué significa que el software esté bien hecho?

El software de calidad es aquel cuyos programadores saben que funciona de
acuerdo a lo que él cliente pidió, que trabaja de manera adecuada, y que pueden
ajustar de manera fácil. Es software que es fácil de cambiar y que no implica un
alto costo \cite{CMartin2022EthicsManifesto}.

Que el software esté bien hecho significa que es un software que trabaja sin el
riesgo de que ocurra algún comportamiento extraño, o que pueda fallar en
cualquier momento, o se tenga que estar reiniciando cada cierto tiempo, etc.

Ahora, cualquier software que esté conectado a internet es vulnerable de ataques
y usarlo también para hacer ataques.

Ahora, regresando a la pregunta del principio, podemos causar dos tipos de
impacto negativo: daños accidentales y daños intencionales.

\subsection{Daños accidentales}

Son aquellos a los que no nos anticipamos o cuyas fallas fueron omitidas. Por
ejemplo:

\begin{itemize}
\item Problemas de seguridad. Por ejemplo las vulnerabilidade en el software.
\item Bugs, comportamientos no esperados: El sistema se cierra cada que el
usuario ejecuta una acción.
\item Algoritmos incorrectos: Nos dieron una especificación que no implementamos
correctamente y eso produjo un fallo. Por ejemplo activar un servicio a un
usuario por el cual no ha pagado. Por ejemplo el bug del 2000, muchos
programadores que no pensaron que su código fuera a llegar hasta ese año.
\end{itemize}

\subsection{Daños intencionales}

Son reglas de negocio que hacen daños a propósito. Algunos ejemplos son los
siguientes:

\begin{itemize}
\item Desición consciente de la gerencia de una empresa, de un equipo de
ingeniería o de ambas. Por ejemplo el
\href{https://repositorio.comillas.edu/rest/bitstreams/295635/retrieve}{caso
Volkswagen}, donde la empresa fabricó autos que eludían la norma estadounidense
de emisión de gases.
\item Estos daños tienen responsabilidad penal.
\item Privacidad de datos. Por ejemplo la
\href{https://elpais.com/mexico/2022-10-01/una-masiva-filtracion-expone-el-poder-del-ejercito-mexicano-en-la-vida-publica.html}{filtración
de datos de la Secretaría de Defensa Nacional} en septiembre de 2022, por un
grupo de hackers llamados \textit{los guacamayos}.
\item
\href{https://www.netflix.com/es/title/81254224#:~:text=Este%20documental%20dramatizado%20analiza%20la,las%20herramientas%20creadas%20por%20ellos.&text=Ve%20todo%20lo%20que%20quieras.&text=De%20Jeff%20Orlowski%2C%20director%20del,'%2C%20ganador%20del%20premio%20Emmy.}
{El dilema de las redes sociales}. Los algoritmos de las redes sociales
están diseñados para que la gente pase más tiempo en ellas, ¿alguien debería de
regular esto?.
\end{itemize}

\subsection{Otros casos de estudio}

Otro caso de estudio de falla en sfotware, fue el caso del helicóptero Chinook
ZD576. Este se estrelló en Mull of Kintyre el viernes 2 de junio de 1994,  y
mató a 29 personas. El miércoles 6 de febrero de 2002, el informe del comité de
la Cámara de los Lores encontró que había evidencia que sugería que errores de
software y hardware habían causado el accidente y no un error del piloto como se
había proclamado originalmente en la investigación oficial. El informe afirmaba
que "está claro que en el momento del accidente todavía había problemas sin
resolver en relación con el sistema Fadec [software] de los Chinook MK2", y
"consideramos que las conclusiones de Boeing [basadas en sus simulaciones por
computadora] no pueden ser considerado exacto” \cite{collins2002software}. Se
llega a la conclusión de que los procedimientos de prueba utilizados no eran lo
suficientemente rigurosos.


\section{¿Ética en la ingeniería de software?}

En el ámbito de la ingeniería de software, la ética tiene lugar cuando cualquier
decisión tomada por los profesionales del área, durante el diseño, desarrollo,
construcción y mantenimiento de los productos de software afecta a otras
personas de manera positiva o negativa. Estas decisiónes pueden ser tomadas por
individuos, equipos, gerencia o la profesión.

El software tiene presencia en la vida diaria de la mayor parte de las personas
en todo el mundo, por este motivo debemos evaluar el impacto que nosotros y
nuestras acciones como profesionales de la ingeniería de software tienen sobre
la sociedad. Algunas personas dependen del software y esto quiere decir que
dependen de programadores. Es entonces cuando surge la necesidad de regularizar
los estándares del área.

Existen diferentes instituciones que se han preocupado por establecer un
estándar a seguir para las personas del área. Estas instituciones han
desarrollado códigos de ética, los cuales sirven para orientar las decisiones
éticas y profesionales de los ingenieros de software. Estos códigos establecen
declaraciones acerca de la interacción de la tecnología y los valores.

El objetivo de estos códigos de ética es asegurar que los profesionales del área
protejan los valores humanos en lugar de dañarlos. Algunas de las instituciones
que regulan la ética en el área de la ingeniería de software son las siguientes:

\begin{itemize}
    \item \href{https://www.sei.cmu.edu/}{Software Engineering Institute (SEI)}
    \item \href{https://www.acm.org/}{Association for Computing Machinery (ACM)}
\item \href{https://www.bcs.org/}{The Chartered Institute for IT (BSC)} (antes
British Computer Society)
\item \href{https://www.ieee.org/}{Institute of Electrical and Electronics
Engineers (IEEE)} \cite{rogerson2002software}
    \item \href{https://russoft.org/}{Russian Software Developer Association (RUS SOFT)}
    \item \href{https://sse.rit.edu/}{Society of Software Engineers (SSE)}
\end{itemize}

Uno de los esfuerzos más importantes para crear un código de ética para los
profesionales de la ingeniería de software, ha sido el realizado por el
\textit{IEEE/ACM} en 1996 \cite{gotterbarn2001software}, en el \textit{Software
Engineering Code of Ethics and Professional Conduct (SECODE)}. Los objetivos de este
código fueron:

\begin{enumerate}
    \item Adoptar definiciones estándar
    \item Definir el conjunto de conocimientos requeridos y las prácticas recomendadas.
    \item Definir estándares éticos
    \item Definir los currículos educativos para (a) Universidad, (b) Posgrado (MS) y (c)
educación continua (Gotterbarnb, 1996).
\end{enumerate}

El \textit{SECODE} contiene ocho principios relacionados al comportamiendo y
desiciones hechas por ingenieros de software profesionales. El Código prescribe
estos principios como obligaciones de cualquier persona que afirme ser o aspire
a ser ingeniero de software \cite{vallor2015introduction}.

Los principios estan resumidos de la siguiente manera:

\begin{enumerate}
\item \textbf{Público:} Los ingenieros de software deben actuar consistentemente
con el interés público.
\item \textbf{Cliente y empleador:} Los ingenieros de software deben actuar en
el mejor de los intereses de su cliente o empleador y que sea consistente con el
interés público.
\item \textbf{Producto:} Los ingenieros de software deben asegurar que sus
productos y modificaciones reúnan los estándares profesionales más altos
posibles.
\item \textbf{Criterio:} Los ingenieros de software deben mantener integridad e
independencia en su juicio profesional.
\item \textbf{Administración:} Los líderes e ingenieros de software promoveran
un enfoque ético para la gestión del desarrollo y mantenimiento del software.
\item \textbf{Profesión:} Los ingenieros de software deben promover la
integridad y reputación de la profesión de acuerdo con el interés público.
\item \textbf{Colegas:} Los ingenieros de software deben ser justos y solidarios
con sus colegas.
\item \textbf{Uno mismo:} Los ingenieros de software participarán en el
aprendizaje permanente en relación con la práctica de su profesión y promoverás
un enfoque ético para la práctica de la profesión.
\end{enumerate}


El Código de Ética y Práctica Profesional de Ingeniería de Software tiene una
serie de cláusulas relevantes para las pruebas de software
\cite{rogerson2002software}. Tres cláusulas indicativas establecen lo siguiente:

\textbf{1.03}. Aprueban el software solo si tienen la creencia bien fundada de que es
seguro, cumple con las especificaciones, pasa las pruebas adecuadas y no
disminuye la calidad de vida, la privacidad ni daña el medio ambiente. El efecto
último de la obra debe ser el bien público.

\textbf{3.10}. Garantizar pruebas, depuración y revisión adecuadas del software
y los documentos relacionados en los que trabajan.

\textbf{3.11}. Garantizar la documentación adecuada, incluidos los problemas
significativos descubiertos y las soluciones adoptadas, para cualquier proyecto
en el que trabajen. \\
\\

Por otro lado, Gotterbarnb \cite{gotterbarn2001software}, plantea que existen
diferentes problemas éticos durante las fases de desarrollo de software, por lo
que las reglas morales varían dependiendo de las diferentes etapas del ciclo de
vida del desarrollo.

Por ejemplo, durante la fase de obtención de requerimientos del software, el
consentimiento informado o un acuerdo de entendimiento es crucial. Durante las
pruebas, los principios sobre no engañar y no hacer trampa son muy importantes.
Sin embargo, en la fase de prueba, por ejemplo, si se acaban los fondos antes de
que las pruebas se completen y no hay más posibilidad de fondos, podemos optar
diferentes opciones; sin embargo, cualquier decisión que tomemos debe estar
condicionada por reglas morales como no engañar al cliente y actuar
profesionalmente. Aunque de acuerdo al tipo de software desarrollado también
pueden entrar reglas como que no cause dolor y que no mate.


\subsection{Beneficios en la adopción de un código de ética en la ingeniería de software.}

Existen beneficios en adoptar un código de ética tanto para ingenieros de
software como para empresas y organizaciones que se dediquen al área. Por
ejemplo, promovemos una cultura de desarrollo de software de calidad, lo cual
atraerá más colegas que deseen participar. Ganaremos confianza pública y
desarrollaremos una imagen profesional que cumple con los estándares de calidad
del software.



\pagebreak

\section{Conclusiones}

\begin{itemize}
\item Dada la enorme cantidad de gente en el mundo que utiliza el software y
que este forma parte de la vida diaria de las personas, es nuestra
responsabilidad como profesionales del desarrollo de software, generar productos
que cubran las necesidades de las personas y evitar que este pueda tener un
impacto negativo sobre la sociedad.

\item El código de ética del \textit{IEEE/ACM} es solo una guía que pretende
orientar a las personas profesionales de la ingeniería de software a tomar
mejores decisiones con el fin de minimizar el impacto negativo que tenemos en la
sociedad.


\end{itemize}

\subsection{¿Qué acciones podemos tomar como ingenieros de software?}

Algunas acciones que podríamos llevar a cabo de manera personal y organizacional
son:

\begin{itemize}
\item Mejorar nuestras prácticas de ingeniería: Seguridad informática, trabajo
en equipo, planificación, pruebas de software automatizadad, calidad del código
y arquitectura, escalabilidad.
\item Disciplina, profesionalismo, mejora continua.
\item Estudio de la moral, y ética profesional.
\item Comprender y evaluar el impacto de nuestras acciones como ingenieros de
software en la sociedad y la economía.
\end{itemize}

\pagebreak
\bibliography{../references/references.bib} 
\bibliographystyle{unsrt}

\end{multicols}
\end{document}
