\documentclass[twoside]{article}

\usepackage{lipsum} % Package to generate dummy text throughout this template

\usepackage[sc]{mathpazo} % Use the Palatino font
\usepackage[T1]{fontenc} % Use 8-bit encoding that has 256 glyphs
\linespread{1.05} % Line spacing - Palatino needs more space between lines
\usepackage{microtype} % Slightly tweak font spacing for aesthetics

\usepackage[hmarginratio=1:1,top=32mm,columnsep=20pt]{geometry} % Document margins
\usepackage{multicol} % Used for the two-column layout of the document
\usepackage[hang, small,labelfont=bf,up,textfont=it,up]{caption} % Custom captions under/above floats in tables or figures
\usepackage{booktabs} % Horizontal rules in tables
\usepackage{float} % Required for tables and figures in the multi-column environment - they need to be placed in specific locations with the [H] (e.g. \begin{table}[H])
\usepackage{hyperref}    % Vinculos [que no estén subrayados]
\hypersetup{
    colorlinks=true,
    linkcolor=magenta,
    urlcolor=blue,
    citecolor=black
    }

\usepackage{lettrine} % The lettrine is the first en\usepackage[hidelinks]{hyperref}    % Vinculos [que no estén subrayados]larged letter at the beginning of the text
\usepackage{paralist} % Used for the compactitem environment which makes bullet points with less space between them

\usepackage{abstract} % Allows abstract customization
\renewcommand{\abstractnamefont}{\normalfont\bfseries} % Set the "Abstract" text to bold
\renewcommand{\abstracttextfont}{\normalfont\small\itshape} % Set the abstract itself to small italic text

\usepackage{titlesec} % Allows customization of titles
\renewcommand\thesection{\Roman{section}} % Roman numerals for the sections
\renewcommand\thesubsection{\Roman{subsection}} % Roman numerals for subsections
\titleformat{\section}[block]{\large\scshape\centering}{\thesection.}{1em}{} % Change the look of the section titles
\titleformat{\subsection}[block]{\large}{\thesubsection.}{1em}{} % Change the look of the section titles

\usepackage{fancyhdr} % Headers and footers
\pagestyle{fancy} % All pages have headers and footers
\fancyhead{} % Blank out the default header
\fancyfoot{} % Blank out the default footer
\fancyhead[C]{Ética en la ingeniería de Software $\bullet$ Noviembre 2022}
\fancyfoot[RO,LE]{\thepage} % Custom footer text

%----------------------------------------------------------------------------------------
%	TITLE SECTION
%----------------------------------------------------------------------------------------

\title{\vspace{-15mm}\fontsize{24pt}{10pt}\selectfont\textbf{Ética en la
ingeniería de Software}}

\author{
\large
\textsc{Ana Maritza Bello Yáñez} \\
\normalsize Centro de Investigación en Computación \\ 
\normalsize {abelloy2022@cic.ipn.mx}
\vspace{-5mm}
}
\date{}

%----------------------------------------------------------------------------------------

\begin{document}

\maketitle % Insert title

\thispagestyle{fancy} % All pages have headers and footers

%----------------------------------------------------------------------------------------
%	ABSTRACT
%----------------------------------------------------------------------------------------

\begin{abstract}

\noindent Este trabajo aborda aspectos relacionados a la ética en la Ingeniería
de Software. También muestra algunos esfuerzos realizados por diferentes
instituciones por definir un estándar de ética para profesionistas de la
ingeniería de software.

\end{abstract}

%----------------------------------------------------------------------------------------
%	ARTICLE CONTENTS
%----------------------------------------------------------------------------------------

\begin{multicols}{2} % Two-column layout throughout the main article text

\section{Introducción}

La ingeniería de software es una disciplina que combina las ciencias de la
computación, ciencias aplicadas y ciencias básicas basandose en la ingeniería.
Esta disciplina busca apoyar al desarrollo de software profesional. Cuando
hablamos de software no nos referimos solamente a programas, sino a toda la
documentación  asociada y a los datos de configuración requerido para hacer que
los programas funcionen de manera correcta \cite{sommerville2005ingenieria}.

La ética profesional, a su vez, consiste en un conjunto de saberes, creencias,
normas y valores que rigen el actuar de las personas en el campo profesional, y
en este caso dentro del campo de desarrollo de software
\cite{yuren2013etica}.

Así al hablar, de ética en el desarrollo de software, estamos considerando todos
aquellos aspectos relacionados a como desempeñamos nuestra profesión como
ingenieros de software.

\section{¿Qué es la ingeniería de software?}

La ingeniería de software, es una disciplina de la ingeniería que se interesa
por todos los aspectos de la producción de software, con el objetivo de que el
producto final (programas de cómputo y documentación asociada) tenga un
funcionamiento correcto. Es decir, que se le entregue al usuario un software con
la funcionalidad y desempeño requeridos, además de que sea confiable y
utilizable.

\section{¿Para qué sirve la ética en la ingeniería de software?}

En el ámbito de la ingeniería de software, la ética tiene lugar cuando cualquier
decisión tomada por los profesionales del área, durante el diseño, desarrollo,
construcción y mantenimiento de los productos de software afecta a otras
personas de manera positiva o negativa. Estas decisiónes pueden ser tomadas por
individuos, equipos, gerencia o la profesión.

El software tiene presencia en la vida diaria de la mayor parte de las personas
en todo el mundo, por este motivo debemos evaluar el impacto que nosotros y
nuestras acciones como profesionales de la ingeniería de software tienen sobre
la sociedad.

Existen diferentes instituciones que se han preocupado por establecer un
estándar a seguir para las personas del área. Estas instituciones han
desarrollado códigos de ética, los cuales sirven para orientar las la decisiones
éticas y profesional de los ingenieros de software. Estos códigos establecen
declaraciones acerca de la interacción de la tecnología y los valores.

El objetivo de estos códigos de ética es asegurar que los profesionales del área
protejan los valores humanos en lugar de dañarlos.

Algunas de las instituciones que regulan la ética en el área de la ingeniería de
software son las siguientes:

\begin{itemize}
    \item \href{https://www.sei.cmu.edu/}{Software Enginerring Institute (SEI)}
    \item \href{https://www.acm.org/}{Association for Computing Machinery (ACM)}
    \item \href{https://www.bcs.org/}{The Chartered Institute for IT (BSC)} (antes British Computer Society)
    \item \href{https://www.ieee.org/}{Institute of Electrical and Electronics Engineers (IEEE)}
    \item \href{https://russoft.org/}{Russian Software Developer Association (RUS SOFT)}
    \item \href{https://sse.rit.edu/}{Society of Software Engineers}
\end{itemize}



\section{¿Dónde y como aplicar la ética en la ingeniería de software?}


\section{Conclusiones}

\begin{itemize}
\item Dada la enorma cantidad de gente en el mundo que utiliza el software y
que este forma parte de la vida diaria de las personas, es nuestra
responsabilidad como profesionales del desarrollo de software, generar productos
que cubran las necesidades de las personas y evitar que este pueda tener un
impacto negativo sobre la sociedad.

\end{itemize}


\pagebreak
\bibliography{../references/references.bib} 
\bibliographystyle{unsrt}

\end{multicols}
\end{document}
